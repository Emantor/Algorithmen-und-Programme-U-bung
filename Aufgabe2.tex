\section{Übung 2 (10.11.2011)}
\subsection{Vergleichs- und Verknüpfungsoperatoren}
\begin{table}[h]
\begin{tabular}{c|c}
Operator & Bedeutung \\
\hline
< & kleiner \\
> & größer \\
>= & größer gleich \\
== & gleich \\
!= & ungleich \\
\&\& & logisches UND \\
|| & logisches ODER \\
\end{tabular}
\end{table}
Nutzung z.B. in \texttt{if} Abfragen
\begin{lstlisting}
if (Bedingung) // WENN Bedingung erfuellt
{
	Anweisung; // fuehre Anweisung aus
}

if (3 > 5)
{
	cout >> "PC kaputt" << endl;
}

int a = 6;
if (zahl == a)
{
	cout << "Zahl gleich" << a << endl;
}
else		// wird ausgefuehrt wenn die Bedingung nicht erfuellt ist.
{
	cout << "Zahl ist ungleich" << a << endl;
}

int b = 9;
if (zahl > a && zahl > b)
{
	cout << "Zahl ist am Groessten" << endl;
}

bool bigger;
bigger = zahl > b;
if (bigger || zahl < b)
{
	cout << "Zahl ungleich" << b << endl;
}
else
{
	cout << "Zahl gleich" << b << endl;
}
\end{lstlisting}

\subsection{Definition von Funktionen (Unterprogramme)}
\begin{lstlisting}
Rueckgabetyp Funktionsname(Parameter1,Parameter2,...)
{
	Anweisung;
	return rueckgabewert; 	// muss vom Datentyp Rueckgabetyp sein
}

int Addiere(int a, int b)
{
	int sum;
	sum = a +b;
	return sum;
}

void Ausgabe(int zahl)
{
	cout << "Zahl ist" << zahl << endl;
}
\end{lstlisting}
Funktionen vom Typ void haben keinen Rueckgabewert und somit keine \texttt{return} Anweisung.

Aufbau eines C++-Programms:
\begin{lstlisting}
// Einbinden von Bibliotheken
#include <iostream>
using namespace std; // noetig fuer cout und cin
// Funktionsdefinitionen
int Addiere(int a, int b)
{
	int sum;
	sum = a +b;
	return sum;
}

void Ausgabe(int zahl)
{
	cout << "Zahl ist" << zahl << endl;
}
// Hauptprogramm
int main()
{
	int zahl1 = 4;
	int zahl2 = 5;
	int summe;
	summe = Addiere(zahl1,zahl2);
	Ausgabe(summe);
	return 0; 		// Im Hauptprogramm nicht notwendig
}
\end{lstlisting}
\subsection{Vom Quellcode zum ausführbaren Programm}
Quellcode(Text) $\rightarrow$ kompilieren (Übersetzung in Maschinensprache) $\rightarrow$ Compilerfehler? \\
Falls ja, muss der Quellcode überprüft werden. \\
Falls Nein (syntaktische Korrektheit des Programms) $\rightarrow$ Programm ausführen $\rightarrow$ Laufzeitfehler? \\
Falls ja, neuerliche Bearbeitung des Quellcodes. \\
Falls nein, Programmende \\
(evtl. mit TikZ in Blockdiagramme umsetzen)

\subsection{Schleifen}
Schleifen werden genutzt, um Anweisungen/Blöcke mehrmals auszuführen.


\texttt{for}-Schleife: Verwendung, wenn Anzahl der Durchläufe bekannt
\begin{lstlisting}
for (Startanweisung, Bedingung, Zaehlanweisung)
{
	Anweisung; // fuehre aus, solange Bedingung erfuellt
}

int z = 0;
for (int i = 1, i < 10, i++)
{
	z = z +i;
	cout << z << endl;
}
\end{lstlisting}
\begin{table}[h]
\begin{tabular}{c|c|c}
Schleifendurchlauf & \texttt{i} & \texttt{z}\\
\hline
1 & 1 & 1 \\
2 & 2 & 3 \\
3 & 3 & 6 \\
\dots & \dots & \dots \\
9 & 9 & 36 + 9 = 45 \\
\end{tabular}
\end{table}
Danach Abbruch, weil \texttt{i} nicht mehr kleiner als 10 ist. \\

\texttt{while}-Schleife: Verwendung, wenn Anzahl der Durchläufe nicht bekannt.
\begin{lstlisting}
while (Bedingung)
{
	Anweisung; // SOLANGE WIE Bedingung ERFUELLT
}

int eingabe = 0;
while (engage != 8)
{
	cout << "Bitte 8 eingeben!" << endl;
	cin >> eingabe;
}
cout << "Danke, Du hast " << eingabe << " eingegeben";
\end{lstlisting}