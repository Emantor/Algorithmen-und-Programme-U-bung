\section{Übung 1 (3.11.2011)}
Benutzername: aupvz \\
Passwort: aupvz1112 \\
\subsection{Aufgabe 1}
Terminierend: Algorithmus hält an \\
Determiniert: Algorithmus liefert bei selber Eingabe selbes Ergebnis \\
Deterministisch: Alg. liefert bei selber Eingabe selbes Ergebnis über die selben Zwischenergebnisse \\

\begin{table}[h]
\begin{tabular}{c|c|c|c}
	 ~ & terminiernd & determiniert & deterministisch \\
	a) &  ja & ja & ja\\
	b) Zahl $\notin \{-1,0,1\}$ & ja & nein & nein \\ 
	Zahl $\in \{-1,0,1\}$ & nein & nein & nein \\ 
\end{tabular}
\end{table}
a) Terminierend? \\
\begin{itemize}
\item[=>] Keine Schleifen
\item[=>] Algorithmus hält direkt an
\item[=>] terminierend
\end{itemize}

Determiniert? \\
\begin{itemize}
\item[->] Eindeutig reproduzierbares Ergebnis für sämtliche Eingabewerte? \\
 p $\in \mathbb{Z}$, $q=0$ => erg= -1, q |= 0 => erg = p \% q
 \item[=>] determiniert
 \end{itemize}
 
 deterministisch?
\begin{itemize}
 \item[-] Gibt es Zwischenergebnisse? (falls nein -> deterministisch)
 \item[-] Hier ja: Ergebnis der Abfrage q |= 0
 \item[-] Zwischenergebnis immer gleich bei gleichem q
 \item[=>] deterministisch
\end{itemize}

b) \begin{itemize}
 \item[-] Rechner wählt beliebige Zahl
  \begin{itemize}
   \item[->] Zufallszahl zwischen -10 und 20
   \item[->] nicht determiniert
   \item[->] nicht deterministisch
  \end{itemize}
  \item[-] Trotzdem terminierend?
  \item[-] Bsp für Zahl = 2
  \begin{itemize}
   \item[1.] Zahl = 2
   \item[2.] Zahl = $2*2$ = 4
   \item[3.] Zahl > 10? nein => gehe zu 2
   \item[2.] Zahl = $4*4$ = 16
   \item[3.] Zahl > 10? ja => Ausgabe und Ende
   \item[=>] terminierend
  \end{itemize}
  \item[-] Aber: Abbruchbedingung der Schleife wird nur erreicht, wenn Zahl im Laufe des Alg. wächst, dies gilt nicht für die Zahl $\in \{-1,0,1\}$ => dann nicht terminiert
\end{itemize}

\subsection{Einführung in C++}
Datentypen \\
\begin{table}[h]
\begin{tabular}{c|c|c}
	Datentyp & Bedeutung & Beispiel \\
	int & Integer-Zahl (ganze Zahl) & -2;45 \\
	float & Gleitkommazahl & -4.342;7.543 \\
	char & Character (Zeichen) & 'a'; 'C'; '?'; '7' \\
	bool & \glq Wahr-Falsch-Variable \grq & true, false \\
	string & Zeichenkette & ''Wort'';''2plus 3=5'' \\
\end{tabular}
\end{table}
\underline{Anlegen einer Variable} \\
\begin{lstlisting}
Datentyp Variablenname;
\end{lstlisting}
Wichtig: Jede Anweisung muss mit einem Semikolon abgeschlossen werden.

Bsp: 
\begin{lstlisting}
float kommazahl;
\end{lstlisting}
Zu beachten bei Namensgebung
\begin{itemize}
\item[-] C++ ist ''case sensitiv'', dh. Groß- und Kleinschreibung wird beachtet
 \begin{lstlisting}
 => int zahl; und int Zahl; // sind zwei verschiedene Variablen
 \end{lstlisting}
 \item[-] Keine Sonderzeichen: bool grüner100\euro{}schein; geht nicht!
 \item[-] Keine reservierten Worte: char string;
 \item[-] Keine Nummern am Anfang: int 5teZahl; geht nicht!
\end{itemize}

\subsubsection{Wertezuweisung}
\begin{lstlisting}
int i; // dies ist ein Kommentar
i = 3; // i wird der Wert 3 zugewiesen
int k;
k = i; // k ist ebenfalls 3
k = i + 5; // k = 3+5 = 8
k = k*k; // k = 8*8 = 64
k = k/i; // k = 64/3 = 21 Achtung:ganzzahlige Division
k = k*(k+2); // k = 3*(21+2) = 69
k = k%50; // k = 69%50 = 19
float f;
f = 1;
f = 1/3; // f=0.33333...
char c;
c = '?';
string s;
s = "Text";
s = s + "zeile"; // s = "Textzeile"
bool b;
b = true;
b = !b; // Negation von b => b = nicht true => b = false
\end{lstlisting}
\subsubsection{Gültigkeitsbereich von Variablen}
\begin{itemize}
\item[-] Bereiche werden durch \{\} gekennzeichnet
\item[-] Variablen sind nur innerhalb desjenigen Bereichs gültig, in dem sie deklariert wurden
\item[-] außerhalb jeden Bereichs deklarierte Variablen heißen \underline{global} und sind überall verfügbar
\item[-] Variablen sind nur NACH ihrer Deklaration verfügbar
\end{itemize}
Bsp:
\begin{lstlisting}
int weltweit = 4; //global
{
	int a;
	{
		a = 1;
		int b = 2;
		a = weltweit + b;
	}
	int c;
	c = a + weltweit;
	c = a + b; //UNGUELTIG!, da b in diesem Bereich nicht bekannt ist
}
\end{lstlisting}
\subsubsection{Ein- und Ausgabe in der Konsole}
\begin{itemize}
\item[-] Nutzung der Funktionen cout (Ausgabe) und cin (Eingabe)
\item[-] Einbindung von <iostream> nötig
\item[-] Befehle müssen \lstinline{using namespace std;} freigeschaltet werden
\item[-] Werte werden mittels \lstinline{<<} und \lstinline{>>} übergeben
\end{itemize}
Bsp:
\begin{lstlisting}
int zahl;
cout << "Bitte Zahl eingeben:";
cin >> zahl;
cout << "Das Quadrat lautet" << zahl*zahl << endl; // endl -> Zeilenumbruch
\end{lstlisting}