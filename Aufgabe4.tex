\section{Aufgabe 4}
\begin{enumerate}
\item Funktionsentwicklung
\begin{lstlisting}
void Mittelwert(float &mittel)
{
	float summe = 0;
	for(int i=0; i<n; i=i+1)
	{
		summe = summe + Notenspiegel[i];
	}
	mittel = summe / N;
}
\end{lstlisting}
Bucketsort für Klausurnoten \\
Wertebereich von 0\dots 6 \\
$\Rightarrow$ 7 Eimer werden benötigt \\
z.B. 3,6,3,1,1,0,3,4,2
\begin{table}[h]
\begin{center}
\begin{tabular}{c|c|c|c|c|c|c|c}
bucket & 0 & 1 & 2 & 3 & 4 & 5 & 6\\
\hline
~& | & || & | & ||| & | & ~ & | \\
\end{tabular}
\end{center}
\end{table}
Hilfsmittel:
\begin{itemize}
\item[-] Array aus Integern (dienen als Eimer)
\item[-] Schleife zum Durchlaufen der Eingabewerte und zum Füllen der Eimer. \\
	Der Zahlenwert entspricht der Position im Array
\item[-] Schleife um die Zahlen sortiert ins Array zurückzuschreiben.
\end{itemize}

\end{enumerate}

