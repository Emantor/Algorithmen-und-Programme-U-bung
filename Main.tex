\documentclass[12pt,a4paper]{scrartcl}
\usepackage{geometry}                % See geometry.pdf to learn the layout options. There are lots.
\geometry{a4paper}                   % ... or a4paper or a5paper or ... 
%\geometry{landscape}                % Activate for for rotated page geometry
%\usepackage[parfill]{parskip}    % A-bombctivate to begin paragraphs with an empty line rather than an indent
\usepackage{graphicx}
\usepackage{amssymb}
\usepackage{epstopdf}
%Eigene Packages
\usepackage{ngerman}			% Deutsch als Sprache (Silbentrennung etc)
\usepackage{longtable}			% Unterstützung für Seitenumfassende Tabellen
\usepackage{gensymb}			% Symbole
\usepackage{ae}				% Besseres Schriftbild in PDF Dateien
\usepackage[pdftex]{hyperref}		% Paket für klickbares Inhaltsverzeichnis, Miniaturen usw
\usepackage[utf8]{inputenc}	% Paket für Umlaute (utf8 encoding), alte haben unter Umständen noch applemac! oder noch schlimmer: westeuropäisch windows
\usepackage [T1]{fontenc}		
% eigene Packages

% Pakete von Rouven
\usepackage{listings}
\usepackage{color}
\usepackage{eurosym}

%color Definitionen
\definecolor{middlegray}{rgb}{0.5,0.5,0.5}
\definecolor{lightgray}{rgb}{0.8,0.8,0.8}
\definecolor{orange}{rgb}{0.8,0.3,0.3}
\definecolor{yac}{rgb}{0.6,0.6,0.1}
 
% Listings Settings
 \lstset{
   language=c++,
   backgroundcolor=\color{lightgray},
   basicstyle=\scriptsize\ttfamily,
   keywordstyle=\bfseries\ttfamily\color{orange},
   stringstyle=\color{blue}\ttfamily,
   commentstyle=\color{middlegray}\ttfamily,
   emph={square}, 
   emphstyle=\color{blue}\texttt,
   emph={[2]root,base},
   emphstyle={[2]\color{yac}\texttt},
   showstringspaces=false,
   flexiblecolumns=false,
   tabsize=2,
   numbers=none,
   numberstyle=\tiny,
   numberblanklines=false,
   stepnumber=1,
   numbersep=10pt,
   xleftmargin=15pt
 }
 
 \hypersetup
{%
pdftitle = {Algorithmen und Programme Übung},
pdfsubject = {AuP},
pdfauthor = {Rouven Czerwinski},
pdfkeywords = {AuP, Algorithmen, Programme, Übung},
colorlinks = {true},
linkcolor={blue},
anchorcolor={black}
}
% Eigene Befehle
\newcommand\txtunderbrace[2]{%
  \settowidth{\myx}{#2}%
  \begin{tabular}[t]{@{}>{\centering}p{\myx}@{}}%
    \ensuremath{\underbrace{\text{#2}}}\tabularnewline
    \makebox[0pt][c]{#1}~\tabularnewline
  \end{tabular}%
}%\txtunderbrace{Text unter der Klammer}{Text im Kontext}

\DeclareGraphicsRule{.tif}{png}{.png}{`convert #1 `dirname #1`/`basename #1 .tif`.png}

\title{Algorithmen und Programme \\ Übung}
\author{Protokolliert von Rouven Czerwinski}
\date{Version vom \today}                                           % Activate to display a given date or no date

\begin{document}
\maketitle
\newpage

\tableofcontents
\listoftables
\newpage

\section{Übung 1 (3.11.2011)}
Benutzername: aupvz \\
Passwort: aupvz1112 \\
\subsection{Aufgabe 1}
Terminierend: Algorithmus hält an \\
Determiniert: Algorithmus liefert bei selber Eingabe selbes Ergebnis \\
Deterministisch: Alg. liefert bei selber Eingabe selbes Ergebnis über die selben Zwischenergebnisse \\

\begin{table}[h]
\begin{tabular}{c|c|c|c}
	 ~ & terminiernd & determiniert & deterministisch \\
	a) &  ja & ja & ja\\
	b) Zahl $\notin \{-1,0,1\}$ & ja & nein & nein \\ 
	Zahl $\in \{-1,0,1\}$ & nein & nein & nein \\ 
\end{tabular}
\end{table}
a) Terminierend? \\
\begin{itemize}
\item[=>] Keine Schleifen
\item[=>] Algorithmus hält direkt an
\item[=>] terminierend
\end{itemize}

Determiniert? \\
\begin{itemize}
\item[->] Eindeutig reproduzierbares Ergebnis für sämtliche Eingabewerte? \\
 p $\in \mathbb{Z}$, $q=0$ => erg= -1, q |= 0 => erg = p \% q
 \item[=>] determiniert
 \end{itemize}
 
 deterministisch?
\begin{itemize}
 \item[-] Gibt es Zwischenergebnisse? (falls nein -> deterministisch)
 \item[-] Hier ja: Ergebnis der Abfrage q |= 0
 \item[-] Zwischenergebnis immer gleich bei gleichem q
 \item[=>] deterministisch
\end{itemize}

b) \begin{itemize}
 \item[-] Rechner wählt beliebige Zahl
  \begin{itemize}
   \item[->] Zufallszahl zwischen -10 und 20
   \item[->] nicht determiniert
   \item[->] nicht deterministisch
  \end{itemize}
  \item[-] Trotzdem terminierend?
  \item[-] Bsp für Zahl = 2
  \begin{itemize}
   \item[1.] Zahl = 2
   \item[2.] Zahl = $2*2$ = 4
   \item[3.] Zahl > 10? nein => gehe zu 2
   \item[2.] Zahl = $4*4$ = 16
   \item[3.] Zahl > 10? ja => Ausgabe und Ende
   \item[=>] terminierend
  \end{itemize}
  \item[-] Aber: Abbruchbedingung der Schleife wird nur erreicht, wenn Zahl im Laufe des Alg. wächst, dies gilt nicht für die Zahl $\in \{-1,0,1\}$ => dann nicht terminiert
\end{itemize}

\subsection{Einführung in C++}
Datentypen \\
\begin{table}[h]
\begin{tabular}{c|c|c}
	Datentyp & Bedeutung & Beispiel \\
	int & Integer-Zahl (ganze Zahl) & -2;45 \\
	float & Gleitkommazahl & -4.342;7.543 \\
	char & Character (Zeichen) & 'a'; 'C'; '?'; '7' \\
	bool & \glq Wahr-Falsch-Variable \grq & true, false \\
	string & Zeichenkette & ''Wort'';''2plus 3=5'' \\
\end{tabular}
\end{table}
\underline{Anlegen einer Variable} \\
\begin{lstlisting}
Datentyp Variablenname;
\end{lstlisting}
Wichtig: Jede Anweisung muss mit einem Semikolon abgeschlossen werden.

Bsp: 
\begin{lstlisting}
float kommazahl;
\end{lstlisting}
Zu beachten bei Namensgebung
\begin{itemize}
\item[-] C++ ist ''case sensitiv'', dh. Groß- und Kleinschreibung wird beachtet
 \begin{lstlisting}
 => int zahl; und int Zahl; // sind zwei verschiedene Variablen
 \end{lstlisting}
 \item[-] Keine Sonderzeichen: bool grüner100\euro{}schein; geht nicht!
 \item[-] Keine reservierten Worte: char string;
 \item[-] Keine Nummern am Anfang: int 5teZahl; geht nicht!
\end{itemize}

\subsubsection{Wertezuweisung}
\begin{lstlisting}
int i; // dies ist ein Kommentar
i = 3; // i wird der Wert 3 zugewiesen
int k;
k = i; // k ist ebenfalls 3
k = i + 5; // k = 3+5 = 8
k = k*k; // k = 8*8 = 64
k = k/i; // k = 64/3 = 21 Achtung:ganzzahlige Division
k = k*(k+2); // k = 3*(21+2) = 69
k = k%50; // k = 69%50 = 19
float f;
f = 1;
f = 1/3; // f=0.33333...
char c;
c = '?';
string s;
s = "Text";
s = s + "zeile"; // s = "Textzeile"
bool b;
b = true;
b = !b; // Negation von b => b = nicht true => b = false
\end{lstlisting}
\subsubsection{Gültigkeitsbereich von Variablen}
\begin{itemize}
\item[-] Bereiche werden durch \{\} gekennzeichnet
\item[-] Variablen sind nur innerhalb desjenigen Bereichs gültig, in dem sie deklariert wurden
\item[-] außerhalb jeden Bereichs deklarierte Variablen heißen \underline{global} und sind überall verfügbar
\item[-] Variablen sind nur NACH ihrer Deklaration verfügbar
\end{itemize}
Bsp:
\begin{lstlisting}
int weltweit = 4; //global
{
	int a;
	{
		a = 1;
		int b = 2;
		a = weltweit + b;
	}
	int c;
	c = a + weltweit;
	c = a + b; //UNGUELTIG!, da b in diesem Bereich nicht bekannt ist
}
\end{lstlisting}
\subsubsection{Ein- und Ausgabe in der Konsole}
\begin{itemize}
\item[-] Nutzung der Funktionen cout (Ausgabe) und cin (Eingabe)
\item[-] Einbindung von <iostream> nötig
\item[-] Befehle müssen \lstinline{using namespace std;} freigeschaltet werden
\item[-] Werte werden mittels \lstinline{<<} und \lstinline{>>} übergeben
\end{itemize}
Bsp:
\begin{lstlisting}
int zahl;
cout << "Bitte Zahl eingeben:";
cin >> zahl;
cout << "Das Quadrat lautet" << zahl*zahl << endl; // endl -> Zeilenumbruch
\end{lstlisting}
\newpage
\section{Übung 2 (10.11.2011)}
\subsection{Vergleichs- und Verknüpfungsoperatoren}
\begin{table}[h]
\begin{tabular}{c|c}
Operator & Bedeutung \\
\hline
< & kleiner \\
> & größer \\
>= & größer gleich \\
== & gleich \\
!= & ungleich \\
\&\& & logisches UND \\
|| & logisches ODER \\
\end{tabular}
\end{table}
Nutzung z.B. in \texttt{if} Abfragen
\begin{lstlisting}
if (Bedingung) // WENN Bedingung erfuellt
{
	Anweisung; // fuehre Anweisung aus
}

if (3 > 5)
{
	cout >> "PC kaputt" << endl;
}

int a = 6;
if (zahl == a)
{
	cout << "Zahl gleich" << a << endl;
}
else		// wird ausgefuehrt wenn die Bedingung nicht erfuellt ist.
{
	cout << "Zahl ist ungleich" << a << endl;
}

int b = 9;
if (zahl > a && zahl > b)
{
	cout << "Zahl ist am Groessten" << endl;
}

bool bigger;
bigger = zahl > b;
if (bigger || zahl < b)
{
	cout << "Zahl ungleich" << b << endl;
}
else
{
	cout << "Zahl gleich" << b << endl;
}
\end{lstlisting}

\subsection{Definition von Funktionen (Unterprogramme)}
\begin{lstlisting}
Rueckgabetyp Funktionsname(Parameter1,Parameter2,...)
{
	Anweisung;
	return rueckgabewert; 	// muss vom Datentyp Rueckgabetyp sein
}

int Addiere(int a, int b)
{
	int sum;
	sum = a +b;
	return sum;
}

void Ausgabe(int zahl)
{
	cout << "Zahl ist" << zahl << endl;
}
\end{lstlisting}
Funktionen vom Typ void haben keinen Rueckgabewert und somit keine \texttt{return} Anweisung.

Aufbau eines C++-Programms:
\begin{lstlisting}
// Einbinden von Bibliotheken
#include <iostream>
using namespace std; // noetig fuer cout und cin
// Funktionsdefinitionen
int Addiere(int a, int b)
{
	int sum;
	sum = a +b;
	return sum;
}

void Ausgabe(int zahl)
{
	cout << "Zahl ist" << zahl << endl;
}
// Hauptprogramm
int main()
{
	int zahl1 = 4;
	int zahl2 = 5;
	int summe;
	summe = Addiere(zahl1,zahl2);
	Ausgabe(summe);
	return 0; 		// Im Hauptprogramm nicht notwendig
}
\end{lstlisting}
\subsection{Vom Quellcode zum ausführbaren Programm}
Quellcode(Text) $\rightarrow$ kompilieren (Übersetzung in Maschinensprache) $\rightarrow$ Compilerfehler? \\
Falls ja, muss der Quellcode überprüft werden. \\
Falls Nein (syntaktische Korrektheit des Programms) $\rightarrow$ Programm ausführen $\rightarrow$ Laufzeitfehler? \\
Falls ja, neuerliche Bearbeitung des Quellcodes. \\
Falls nein, Programmende \\
(evtl. mit TikZ in Blockdiagramme umsetzen)

\subsection{Schleifen}
Schleifen werden genutzt, um Anweisungen/Blöcke mehrmals auszuführen.


\texttt{for}-Schleife: Verwendung, wenn Anzahl der Durchläufe bekannt
\begin{lstlisting}
for (Startanweisung, Bedingung, Zaehlanweisung)
{
	Anweisung; // fuehre aus, solange Bedingung erfuellt
}

int z = 0;
for (int i = 1, i < 10, i++)
{
	z = z +i;
	cout << z << endl;
}
\end{lstlisting}
\begin{table}[h]
\begin{tabular}{c|c|c}
Schleifendurchlauf & \texttt{i} & \texttt{z}\\
\hline
1 & 1 & 1 \\
2 & 2 & 3 \\
3 & 3 & 6 \\
\dots & \dots & \dots \\
9 & 9 & 36 + 9 = 45 \\
\end{tabular}
\end{table}
Danach Abbruch, weil \texttt{i} nicht mehr kleiner als 10 ist. \\

\texttt{while}-Schleife: Verwendung, wenn Anzahl der Durchläufe nicht bekannt.
\begin{lstlisting}
while (Bedingung)
{
	Anweisung; // SOLANGE WIE Bedingung ERFUELLT
}

int eingabe = 0;
while (engage != 8)
{
	cout << "Bitte 8 eingeben!" << endl;
	cin >> eingabe;
}
cout << "Danke, Du hast " << eingabe << " eingegeben";
\end{lstlisting}
\newpage
\section{Übung 3 (17.11.2011)}
\begin{table}[h]
\begin{tabular}{c|c|c|c}
Wertebereich & terminierend & determinierend & deterministisch \\
\hline
a > 20 & ja & ja & nein\\
10 < a <= 20 & ja & nein & nein \\
2 <= a <= 10 & ja & ja & ja \\
1 < a < 2 & ja & ja & ja  \\
a <= 1 & nein & nein & nein \\
\end{tabular}
\end{table}
Fallunterscheidungen nötig
\begin{itemize}
\item a > 20:
\begin{itemize}
 \item[-] 1. +2. if-Bedingung erfüllt
 \item[-] \underline{Zufallszahl} wird erst abgezogen, dann wieder hinzuaddiert \\
 $\Rightarrow$ terminierend, determiniert, nicht deterministisch
\end{itemize}
\item 10 < a <= 20:
\begin{itemize}
\item[-] nur 1. if-Bedingung erfüllt
\item[-] Zufallszahl wird hinzuaddiert \\
 $\Rightarrow$ terminierend, nicht determiniert, nicht deterministisch
\end{itemize}
\item 10 <= a:
\begin{itemize}
\item[-] if-Bedingung nicht erfüllt
\item[-] while-Schleife nur für a < 2 \\
 $\Rightarrow$ für 2 <= a <= 10 wird nur der Eingabewert zurückgegeben \\
 $\Rightarrow$ terminierend, determiniert, deterministisch
\end{itemize}
\item 1 < a <2:
\begin{itemize}
 \item[-] a wächst bis a => 2, dann wird Schleife beendet \\
 $\Rightarrow$ terminierend, determiniert, deterministisch
\end{itemize}
\item a <= 1:
\begin{itemize}
 \item[-] a muss wachsen, damit Abbruchbedingung erfüllt wird \\
 $\Rightarrow$ nur für a > 1der Fall \\
  $\Rightarrow$ nicht terminierend, nicht determiniert, nicht deterministisch
\end{itemize}
\end{itemize}

\subsection{Aufgabe 3}
\begin{enumerate}
\item Durchschnitt
\begin{enumerate}
\item[a)] Eingabewerte: Start- und Endwert des Wertebereichs \\
	      Rückgabewerte: Durchschnittswert, ganze Zahl \\
	      $\Rightarrow$ Funktionskopf: int Durchschnitt(int \texttt{start}, int \texttt{end})
\item[b)] Vorgehen:
\begin{enumerate}
\item Aufsummieren aller Zahlen von \texttt{start} bis \texttt{end}
\item Bestimmen der Anzahl der Summanden
\item Teilen der Summe durch die Anzahl
\item Rückgabe des Ergebnisses
\end{enumerate}
Nötige Hilfsmittel:
\begin{itemize}
\item[-] Variable zur Speicherung der Summe
\item[-] Schleife (for oder while)
\item[-] Variable für die Anzahl
\end{itemize}
\item Lösung mittels for-Schleife:
\begin{lstlisting}
int Durchschnitt_FOR(int start, int end)
{
	int summe = 0; // Initialisierung mit 0
	for (int i=start; i<=end; i=i+1)
	{
		summe =summe + i;
	}
	int anzahl = end - start;
	int schnitt = summe / Anzahl;
	return schnitt;
}
\end{lstlisting}
Lösung mittels while-Schleife:
\begin{lstlisting}
int Durchschnitt_WHILE(int start, int end)
{
	int summe = 0; // Initialisierung mit 0
	int anzahl = end - start;

	int i=start;
	while (i<=end)
	{
		summe = summe + i;
		i = i +1;
		Anzahl = Anzahl + 1;
	}
	int schnitt = summe / Anzahl;
	return schnitt;
}
}
\end{lstlisting}
Hauptprogramm:
\begin{lstlisting}
int main()
{
	int ds = Durchschnitt_FOR(4, 9);
	cout << "Durschnitt ist " << ds << endl;
}
\end{lstlisting}
\end{enumerate}
\item Quersumme \\
Quersumme(538) = 5 + 4 + 8 = 16; \\
\begin{enumerate}
\item Problem: Wie können die Ziffern separiert werden? \\
	Lösung: Nutzung des Operators \%10 und /10
\item Vorgehen: \\
	Aufsummieren von hinten: \\
	QS(538) = 8 + QS(53) = 8 + 3 QS(5) = 8 + 3 + 5 QS(0) \\
	Algorithmus: Solang die Zahl größer als Null:
	\begin{enumerate}
	\item Ermitteln und Aufsummieren der letzten Ziffer
	\item Abtrennen des letzten Ziffer
	\end{enumerate}
	Nötige Hilfsmittel:
	\begin{itemize}
	\item Variable für Summe
	\item Schleife (bis Zahl == 0)
	\item Modulo Operator
	\end{itemize}
\item Funktionskopf:
\begin{enumerate}
\item zu übergeben: Ganze Zahl $\Rightarrow$ int zahl
\item Rückgabe: Quersumme $\Rightarrow$ int summe \\
\lstinline{int Quersumme(int zahl)}
\end{enumerate}
\item Programm:
\begin{lstlisting}
int Quersumme(int zahl)
{
	int summe = 0;
	while (zahl > 0)
	{
	int Ziffer = zahl % 10;
	summe = summe +ziffer;
	zahl = zahl / 10;
	}
	return summe;
}
\end{lstlisting}
\newpage

Ablauf
Vor der Schleife: summe = 0, zahl = 538
\begin{table}[h]
\begin{center}
\begin{tabular}{c|c|c|c}
Durchlauf & Ziffer & summe & zahl\\
\hline
1. & 538\%10=8 & 0 + 8 = 8 & 538/10 = 53 \\
2. & 53\%10=3 & 8 + 3 = 11 & 53/10 = 5 \\
3. & 5\%10=5 & 11 + 5 = 16 & 5/10 = 0 \\
\end{tabular}
\end{center}
\end{table}
$\Rightarrow$ Abbruch, da zahl nicht mehr > 0
$\Rightarrow$ Rückgabe der Summe
\end{enumerate}
\end{enumerate}


\end{document}