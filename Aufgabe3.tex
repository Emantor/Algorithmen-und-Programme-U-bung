\section{Übung 3 (17.11.2011)}
\begin{table}[h]
\begin{tabular}{c|c|c|c}
Wertebereich & terminierend & determinierend & deterministisch \\
\hline
a > 20 & ja & ja & nein\\
10 < a <= 20 & ja & nein & nein \\
2 <= a <= 10 & ja & ja & ja \\
1 < a < 2 & ja & ja & ja  \\
a <= 1 & nein & nein & nein \\
\end{tabular}
\end{table}
Fallunterscheidungen nötig
\begin{itemize}
\item a > 20:
\begin{itemize}
 \item[-] 1. +2. if-Bedingung erfüllt
 \item[-] \underline{Zufallszahl} wird erst abgezogen, dann wieder hinzuaddiert \\
 $\Rightarrow$ terminierend, determiniert, nicht deterministisch
\end{itemize}
\item 10 < a <= 20:
\begin{itemize}
\item[-] nur 1. if-Bedingung erfüllt
\item[-] Zufallszahl wird hinzuaddiert \\
 $\Rightarrow$ terminierend, nicht determiniert, nicht deterministisch
\end{itemize}
\item 10 <= a:
\begin{itemize}
\item[-] if-Bedingung nicht erfüllt
\item[-] while-Schleife nur für a < 2 \\
 $\Rightarrow$ für 2 <= a <= 10 wird nur der Eingabewert zurückgegeben \\
 $\Rightarrow$ terminierend, determiniert, deterministisch
\end{itemize}
\item 1 < a <2:
\begin{itemize}
 \item[-] a wächst bis a => 2, dann wird Schleife beendet \\
 $\Rightarrow$ terminierend, determiniert, deterministisch
\end{itemize}
\item a <= 1:
\begin{itemize}
 \item[-] a muss wachsen, damit Abbruchbedingung erfüllt wird \\
 $\Rightarrow$ nur für a > 1der Fall \\
  $\Rightarrow$ nicht terminierend, nicht determiniert, nicht deterministisch
\end{itemize}
\end{itemize}

\subsection{Aufgabe 3}
\begin{enumerate}
\item Durchschnitt
\begin{enumerate}
\item[a)] Eingabewerte: Start- und Endwert des Wertebereichs \\
	      Rückgabewerte: Durchschnittswert, ganze Zahl \\
	      $\Rightarrow$ Funktionskopf: int Durchschnitt(int \texttt{start}, int \texttt{end})
\item[b)] Vorgehen:
\begin{enumerate}
\item Aufsummieren aller Zahlen von \texttt{start} bis \texttt{end}
\item Bestimmen der Anzahl der Summanden
\item Teilen der Summe durch die Anzahl
\item Rückgabe des Ergebnisses
\end{enumerate}
Nötige Hilfsmittel:
\begin{itemize}
\item[-] Variable zur Speicherung der Summe
\item[-] Schleife (for oder while)
\item[-] Variable für die Anzahl
\end{itemize}
\item Lösung mittels for-Schleife:
\begin{lstlisting}
int Durchschnitt_FOR(int start, int end)
{
	int summe = 0; // Initialisierung mit 0
	for (int i=start; i<=end; i=i+1)
	{
		summe =summe + i;
	}
	int anzahl = end - start;
	int schnitt = summe / Anzahl;
	return schnitt;
}
\end{lstlisting}
Lösung mittels while-Schleife:
\begin{lstlisting}
int Durchschnitt_WHILE(int start, int end)
{
	int summe = 0; // Initialisierung mit 0
	int anzahl = end - start;

	int i=start;
	while (i<=end)
	{
		summe = summe + i;
		i = i +1;
		Anzahl = Anzahl + 1;
	}
	int schnitt = summe / Anzahl;
	return schnitt;
}
}
\end{lstlisting}
Hauptprogramm:
\begin{lstlisting}
int main()
{
	int ds = Durchschnitt_FOR(4, 9);
	cout << "Durschnitt ist " << ds << endl;
}
\end{lstlisting}
\end{enumerate}
\item Quersumme \\
Quersumme(538) = 5 + 4 + 8 = 16; \\
\begin{enumerate}
\item Problem: Wie können die Ziffern separiert werden? \\
	Lösung: Nutzung des Operators \%10 und /10
\item Vorgehen: \\
	Aufsummieren von hinten: \\
	QS(538) = 8 + QS(53) = 8 + 3 QS(5) = 8 + 3 + 5 QS(0) \\
	Algorithmus: Solang die Zahl größer als Null:
	\begin{enumerate}
	\item Ermitteln und Aufsummieren der letzten Ziffer
	\item Abtrennen des letzten Ziffer
	\end{enumerate}
	Nötige Hilfsmittel:
	\begin{itemize}
	\item Variable für Summe
	\item Schleife (bis Zahl == 0)
	\item Modulo Operator
	\end{itemize}
\item Funktionskopf:
\begin{enumerate}
\item zu übergeben: Ganze Zahl $\Rightarrow$ int zahl
\item Rückgabe: Quersumme $\Rightarrow$ int summe \\
\lstinline{int Quersumme(int zahl)}
\end{enumerate}
\item Programm:
\begin{lstlisting}
int Quersumme(int zahl)
{
	int summe = 0;
	while (zahl > 0)
	{
	int Ziffer = zahl % 10;
	summe = summe +ziffer;
	zahl = zahl / 10;
	}
	return summe;
}
\end{lstlisting}
\newpage

Ablauf
Vor der Schleife: summe = 0, zahl = 538
\begin{table}[h]
\begin{center}
\begin{tabular}{c|c|c|c}
Durchlauf & Ziffer & summe & zahl\\
\hline
1. & 538\%10=8 & 0 + 8 = 8 & 538/10 = 53 \\
2. & 53\%10=3 & 8 + 3 = 11 & 53/10 = 5 \\
3. & 5\%10=5 & 11 + 5 = 16 & 5/10 = 0 \\
\end{tabular}
\end{center}
\end{table}
$\Rightarrow$ Abbruch, da zahl nicht mehr > 0
$\Rightarrow$ Rückgabe der Summe
\end{enumerate}
\item Potenz
\begin{enumerate}
\item Vorgehen \\
$a^b = a*a* \dots *a$ \\
a muss mehrmals mit sich selbst multipliziert werden \\
$\Rightarrow$ Schleife: Frage: Wie viele Durchläufe? \\
z.B. $a^3 = a * a * a$ \\
2 Multiplikationen \\
$\Rightarrow$ Schleife muss (b-1)-mal durchlaufen werden \\
$\Rightarrow$ fortschleife zur Multiplikation mit Index von 1 bis b-1 \\
Sonderfälle: \\
\begin{itemize}
\item b=0: Ergebnis immer 1 \\
\item b negativ: Ergebnis $\frac{1}{a^{-b}}$ \\
\end{itemize}
Algorithmus:
\begin{enumerate}
\item Prüfe, ob b==0. Falls ja, gib 1 zurück
\item Prüfe, ob b<0. Falls ja, negiere b und merke
\item Multipliziere a (b-1)-mal mit sich selbst
\item Falls b ursprünglich negativ war, bilde Kehrwert
\item Gib das Ergebnis zurück
\end{enumerate}
Nötige Hilfsmittel:
\begin{enumerate}
\item if-Abfrage
\item Hilfsvariable War b negativ (\texttt{bool})
\item fortschleife + Variable für Zwischenergebnis
\end{enumerate}
\item zu übergeben an die Funktion: \\
Basis $\in \mathbb{R}$, Exponent $\in \mathbb{Z}$ \\
$\Rightarrow$ \texttt{float} a, \texttt{int} b \\
Rückgabewert $a^b =\mbox{\texttt{float}}*\mbox{\texttt{float}}*\mbox{\texttt{float}}$ \\
$\Rightarrow$ \texttt{float} Ergebnis \\
Funktionskopf
\begin{lstlisting}
float Potenz(float a, int b)
\end{lstlisting}
\item Funktion
\begin{lstlisting}
float Potenz(float a, int b)
{
	if(b==0)
	{
		return 1;
	}
	bool neg=false;
	
	if(b<0)
	{
		neg = true;
		b = b * (-1);
	}
	float ergebnis = a;
	for (int i=1; i<=b-1; i=i-1)
	{
	ergebnis = ergebnis * a;
	}
	if (neg==true)
	{
		ergebnis = 1 / ergebnis;
	}
	return ergebnis;
}
\end{lstlisting}
\item Hauptprogramm
\begin{lstlisting}
int main()
{
	float basis;
	int exponent;
	cout << "Basis?";
	cin >> basis;
	cout << "Exponent?";
	cin >> exponent;
	float potenz;
	potent = Potenz(basis, exponent);
	cout << basis << " hoch " << Exponent << " = " << potenz << endl;
}
\end{lstlisting}
\end{enumerate}
\end{enumerate}

\subsection{Parameterübergabe an Funktionen}
call-by-value: Werteaufruf, es wird eine lokale Kopie der Variable erzeugt. \\
call-by-reference: Referenzaufruf, es wird eine Referenz ($\Rightarrow$ ein Verweis) auf die Variable Im Hauptprogramm übergeben. Wird in Quellcode durch \glqq \&\grqq  signalisiert.
\begin{lstlisting}
void SquareCube(int &squared, int &cubed, in i)
{
	cubed = i * i * i;
	i = i * i;
	squared = i;
}

int main()
{
	int quadriert, kubiert;
	int zahl = 7;
	SquareCube(quadriert, kubiert, zahl);
	cout << zahl << " hoch 2 = " << quadriert << endl;
	cout << zahl << " hoch 3 = " << kubiert << endl;
}
\end{lstlisting}
\subsection{Arrays}
Arrays sind Vektoren (=Datenfelder) mit mehreren Elementen vom gleichen Datentyp
\begin{lstlisting}
Datentyp Variablenname[Anzahl der Elemente];
\end{lstlisting}
Zugriff auf einzelne Elemente des Vektors über den Index i=0\dots Anzahl-1 \\
z.B.
\begin{lstlisting}
int zahlen[6]; // ein Vektor bestehend aus 6 Zahlen
zahlen[0] = 4; // Wertezuweisung des ersten Elements
zahlen[5] = 23; // Wertezuweisung des letzten Elements
\end{lstlisting}
Mehrdimensionale Arrays:
\begin{lstlisting}
float Matrix[3][4];
\end{lstlisting}
\begin{table}[h]
\begin{center}
\begin{tabular}{c|c|c|c|c}
~ & 0 & 1 & 2 & 3 \\
\hline
0 & 4.76 & ~& ~&~\\
1 & ~& ~& ~&5.1\\
2 & ~& ~& ~&~\\
\end{tabular}
\end{center}
\end{table}
\begin{lstlisting}
Matrix[0][0] = 4.67; // Element in 0. Zeile, 0. Spalte
Matrix[1][4] = 5.1;
\end{lstlisting}